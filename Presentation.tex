\documentclass{beamer}
\usepackage{beamerthemesplit,graphicx,wrapfig, slashed, verbatim,hyperref,color,colortbl,tabularx}
% \usepackage{graphicx,wrapfig, slashed, subfigure,verbatim,hyperref,color,colortbl,tabularx}
\usepackage{amsmath,amssymb}
\usepackage{helvet} %font helvetica
\usepackage{units}
\usepackage{pdftricks}
\usepackage{pstricks}
\usepackage{beamerfoils}
%\usepackage{babel}
\usepackage[utf8]{inputenc}
\usepackage[T1]{fontenc}
\usepackage{ulem}
\usepackage{epstopdf}
\usepackage{color}
\usepackage{feynmf}
\usepackage{subcaption}
%\usepackage{biblatex}
\usepackage[
  locale=DE,
  separate-uncertainty=true,
  decimalsymbol=.,
  per-mode=symbol-or-fraction,
]{siunitx}
\usepackage{multirow}

% \usetheme{Hannover}
\useoutertheme{infolines2}
%\font\ttfAkkuratLightOffice AkkuratLightOffice-Regular at10pt
\colorlet{structure}{green!50!black}
\definecolor{tugreen}{RGB}{132,184,24}
\definecolor{tugrey}{RGB}{178,179,182}
\definecolor{tured}{RGB}{205,0,47}
\setbeamercolor{palette primary}{bg=tugreen,fg=black}
\setbeamercolor{palette secondary}{bg=tugrey!50!tugreen,fg=black}
\setbeamercolor{palette quaternary}{fg=black, bg=tugrey}
\setbeamercolor{caption name}{fg=tugreen}
\setbeamercolor{palette tertiary}{fg=black,bg=tugrey}


\setbeamercolor{palette compare}{bg=white!80!tugreen,fg=black}
\setbeamercolor{palette misc}{bg=white!80!tugreen,fg=black}
\setbeamercolor{palette white}{bg=white!99!black,fg=black}

\setbeamercolor{itemize item}{fg=tugreen}
\setbeamercolor{itemize subitem}{fg=tugreen}
\setbeamercolor{itemize subsubitem}{fg=tugreen}
\setbeamercolor{enumerate item}{fg=tugreen}
\setbeamertemplate{itemize item}[square]
\setbeamercolor{titlelike}{fg=tugreen, bg=white}


\title{Constrains on the top quark width from cross section and branching ratio measurements using EFTfitter}


\author{Salvatore La Cagnina}
\institute[TU Dortmund]{
\scriptsize Technische Universität Dortmund, Lehrstuhl EIV \\ 
\vspace{0.5cm}
}


\date{\today}

\MyLogo{\includegraphics[width=3cm]{images/tud_logo_cmyk.pdf}} 
\setbeamertemplate{navigation symbols}{}


\begin{document}

\frame{\titlepage}

%\section{SECTION}
%\subsection{SUBSECTION}
% \LogoOff
%\begin{frame}[fragile]
%\frametitle{FRAMETITLE}
%\begin{beamercolorbox}[rounded=true,shadow=true]{palette misc}
%One type of colorbox
%\end{beamercolorbox}
%\begin{beamercolorbox}[rounded=true,shadow=true]{palette quaternary}
%Second type of colorbox
%\end{beamercolorbox}
%\end{frame}

\section{General}
\subsection{Structure}
%\begin{frame}
%	\frametitle{Topics} \vspace{-3mm}
%	\begin{beamercolorbox}[rounded=true,shadow=true]{palette misc}
%		\begin{itemize}
%			\item Main objective: calculate a pdf for the top quark width
%			\item Build a model with observables, measurements and parameters
%			\item Implement dependencies of the observables from the parameters
%			\item Include measurements
%			\item EFTfitter is used in order to implement that model and perform the calculation
%		\end{itemize}\vspace{2mm}
%	\end{beamercolorbox}
%\end{frame}
\begin{frame}
	\frametitle{Content} \vspace{-3mm}
	\begin{beamercolorbox}[rounded=true,shadow=true]{palette misc}
		\begin{itemize}
			\item The Standard Model
			\item Top quark measurements at LHC
			\item Inference of model parameters in Bayesian analysis
			\item Implementation of Bayesian analysis in EFTfitter
			\item Combination of measurements with the BLUE method and the EFTfitter 
			\item Constraining the top quark width using EFTfitter
			\item Conclusion and Outlook
		\end{itemize}\vspace{2mm}
	\end{beamercolorbox}
\end{frame}

\section{Standard}
\subsection{Model}
\begin{frame}
	\frametitle{The Standard Model} \vspace{-3mm}
	\input{tables/sm.tex}
	\begin{beamercolorbox}[rounded=true,shadow=true]{palette misc}
		\begin{itemize}
			\item Vector bosons as mediators of fundamental forces
			\item Fermion mass increases per generation
			\item Gravitation is not included
		\end{itemize} \vspace{2mm}
	\end{beamercolorbox}
\end{frame}

\section{Top}
\subsection{Quark}
\begin{frame}
	\vspace{-3mm}
	\frametitle{The Top Quark} \vspace{-3mm}
	\begin{minipage}{0.48\textwidth}
		\begin{figure}
			\centering
			\includegraphics[width=0.6\linewidth]{images/feyngluonfusiontt.pdf}
		\caption{Example process for \texorpdfstring{$t\bar{t}$}a  production at the LHC.}
		\end{figure}
	\end{minipage}
	\hfill
	\begin{minipage}{0.48\textwidth}
		\begin{figure}
			\centering
			\includegraphics[width=0.6\linewidth]{images/feynst.pdf}
		\caption{Example process for single top production at the LHC.}
		\end{figure}
	\end{minipage}
	\vspace{-5mm}
	\begin{beamercolorbox}[rounded=true,shadow=true]{palette misc}
		\begin{itemize}
			\item Up-type quark of third generation
			\item Charge of $Q = \SI{2/3}{e}$
			\item Lager mass then every other particle
			\item Decays before hadronization
			\item Can either be produced in pairs (strong interaction) or single (weak interaction)
		\end{itemize} \vspace{2mm}
	\end{beamercolorbox}
\end{frame}

\section{Bayesian}
\subsection{Analysis}
\begin{frame}
	\frametitle{Inference of model parameters}\vspace{-5mm}
	\begin{beamercolorbox}[rounded=true,shadow=true]{palette misc}
		\begin{itemize}
			%\item Physical Quantity $\vec{x}$
			%\item Parameters $\vec{\lambda}$
			\item A function $g(\vec{x}|\vec{\lambda},M)$ describes physical quantity $\vec{x}$ with parameters $\vec{\lambda}$ in a model $M$ 
			\item Based on Bayes Theorem
		\end{itemize}
		\begin{equation*}
		P(\vec{\lambda},M|\vec{D}) = \frac{P(\vec{D}|\vec{\lambda},M) \, P(\vec{\lambda},M)}{\int \! g(\vec{x}=\vec{D}|\vec{\lambda},M) P_i(\vec{\lambda},M)  d\vec{\lambda}} \, .
		%\label{eqn:Bayes}
		\end{equation*}
		\begin{itemize}
			\item $P(\vec{\lambda},M)$ prior probability distribution
			\item Distribution of single parameter $\vec{\lambda}_i$ with
		\end{itemize}
		\begin{equation*}
		P(\lambda_i|\vec{D}) = \int \prod_{i \neq j} P(\vec{\lambda}|\vec{D}) \, d\lambda_{j} \, .
		%\label{eqn:Marginalized}
		\end{equation*}
	\end{beamercolorbox}
\end{frame}

\section{The}
\subsection{EFTfitter}
\begin{frame}
		\frametitle{The EFTfitter}
		\vspace{-6mm}
	\begin{beamercolorbox}[rounded=true,shadow=true]{palette quaternary}
		\begin{itemize}
		\vspace{-3mm}
		\item Based on BAT (Bayesian Analysis Toolkit) [1]
		\item Individual model implementation
		\item Generic tool for interpretation of measurements in context of EFT
		\item Combines measurements comparable to BLUE method
		\vspace{-1mm}
		\end{itemize}
	\end{beamercolorbox}
	\begin{beamercolorbox}[rounded=true,shadow=true]{palette misc}
		\begin{minipage}[b]{0.80\textwidth}
			Combination of measurements
			\begin{itemize}
			\item Observables $\vec{y}$ = Parameters $\vec{\lambda}$
			\item N number of measurements, n number of observables,$\vec{D}_i$ single measurement
			\item $U_{ik}$ unity if measurement matches with observable
			\end{itemize}
			\end{minipage}\vspace{-3mm}
			\begin{equation*}
				-2 \ln{(P(\vec{D}|\vec{\lambda}))} = \sum_{i=1}^{n} \, \sum_{j=1}^{n} \, \sum_{k=1}^{N} \, \sum_{l=1}^{N} \, [D_i - U_{ik} \lambda_k] \, M_{ij}^{-1} \, [D_j - U_{il} \lambda_l] \, ,
			\label{eqn:min}
			\end{equation*}\vspace{-3mm}
		\hfill
		\begin{minipage}[t]{0.15\textwidth} \vspace{-7.8cm} 
			\hspace{-1cm}
			\includegraphics[width=1.3\textwidth]{images/BATLogo.png}
		\end{minipage}
	\end{beamercolorbox}

		[1] Caldwell, Kollár, Kröninger 2009
		\vspace{-3mm}
\end{frame}

\section{BLUE}
\subsection{Method}
\begin{frame}
\frametitle{Combination of measurements}
\vspace{-5mm}
\begin{beamercolorbox}[rounded=true,shadow=true]{palette misc}
	\begin{itemize}\vspace{-2mm}
		\item BLUE = Best Linear Unbiassed Estimator Method\\
		\hspace{3mm}(Lyons, Gibaut, Clifford 1988)
		%$\bullet$ construct a combination $\hat{y}$ out of multiple mesurements\\
		\item Build linear combination of single measurements $y_i$
		\item Minimize variance by minimizing weighted sum $S$
	\end{itemize}
	\begin{equation*}
		\hat{y} = \sum_i \alpha_i y_i \qquad S = \sum_i \sum_j (y' - y_i) E^{-1}_{ij} (y' - y_j)
	\end{equation*}
	\begin{itemize}\vspace{-3mm}
		\item $E_{ij}$ = Error matrix, $S$ follows $\chi^2$ distribution
	\end{itemize}
\end{beamercolorbox}
\begin{beamercolorbox}[rounded=true,shadow=true]{palette quaternary}
	\begin{itemize}\vspace{-3mm}
		\item In case of no correlation ($E_{ij}$ is diagonal)
	\end{itemize}
	\begin{equation*}
		\hat{y} = \left(\sum_i \frac{y_i}{\sigma_i^2}\right) / \left(\sum_i \frac{1}{\sigma^2_i}\right) \qquad\frac{1}{\sigma^2} = \sum_i \frac{1}{\sigma^2_i}
	\end{equation*}
\end{beamercolorbox}
\end{frame}

\section{Comparison EFTfitter}
\subsection{and BLUE method}
\begin{frame}
	\frametitle{The example: Single top + W boson cross section}\vspace{-2mm}
	\begin{figure}
		\includegraphics[width=0.35\textwidth]{images/feynstwt.pdf}
		\caption{Example process for single top + W boson production.}
	\end{figure}\vspace{-4mm}
	\begin{beamercolorbox}[rounded=true,shadow=true]{palette misc}
	\begin{itemize}
		\item  Example for BLUE for a combination of measurements
		\item  Combination of measurements for the production of a top quark and a W boson in pp collisions at $\sqrt{s} = \SI{8}{\TeV}$ 
		\item  Features two measurements, one from ATLAS and one from CMS
		\item  Combines roughly 20 uncertainties
		\item  From ATLAS-CONF-2014-052
	\end{itemize}\vspace{2mm}
\end{beamercolorbox}
\end{frame}

\begin{frame}
\frametitle{The measured cross sections}\vspace{-3mm}

\begin{beamercolorbox}[rounded=true,shadow=true]{palette misc}\vspace{-2mm}
\begin{itemize}
	\item $\text{ATLAS} : \sigma_{tW} = \SI{27.20 \pm 6.35}{\pico\barn}$
	\item $\text{CMS} : \sigma_{tW} = \SI{23.40 \pm 5.07}{\pico\barn}$
	%\item ATLAS: $\sigma_{tW} = 27.2 \si{\pico\barn}$ and CMS: $\sigma_{tW} = 23.4 \si{\pico\barn}$
	\item First comparison only statistical uncertainties
	\item Tables with uncertainties (values in \si{\pico\barn}):
\end{itemize}
\end{beamercolorbox}

\begin{table}[h]
%\caption{Values of the uncertainties of the ATLAS and CMS measurement of the single top + W boson production including their correlation sorted by category.}
\begin{minipage}[b]{0.48\textwidth}
	\resizebox{\textwidth}{!}{
	\input{tables/messwerte1.tex} }
\end{minipage}
\hfill
\begin{minipage}[b]{0.51\textwidth}
	\resizebox{\textwidth}{!}{
	\input{tables/messwerte2.tex} }
\end{minipage}
\label{tab:ST}
\end{table}
\end{frame}

\begin{frame}
	\frametitle{}
	\begin{figure}[h]
	\centering
		\begin{minipage}[b]{.49\textwidth}
		  \includegraphics[width=1\textwidth]{plots/EFT_1_nocorr.pdf}
		  \caption{Combined measurement using EFTfitter. Corresponding\\ value: $\sigma_{tW} = \SI{25.26 \pm 1.35}{\pico\barn}$}
		\end{minipage}
		%\hfill
		\begin{minipage}[b]{.49\textwidth}
		  \includegraphics[width=1\textwidth]{plots/BLUE_1_nocorr.pdf}
		  \caption{Combined measurement using BLUE. Corresponding\\ value: $\sigma_{tW} = \SI{25.26 \pm 1.35}{\pico\barn}$}
		\end{minipage}
	\end{figure}\vspace{-2mm}
	\begin{beamercolorbox}[rounded=true,shadow=true]{palette misc}\vspace{-2mm}
		\begin{itemize}
			\item No correlation between statistical uncertainties
			\item Same result for both methods
		\end{itemize}
	\end{beamercolorbox}
\end{frame}

\begin{frame}
	\frametitle{}\vspace{-2mm}
	\begin{figure}[h]
	\centering
		\begin{minipage}[b]{.49\textwidth}
		  \includegraphics[width=1\textwidth]{plots/EFT_1_poscorr.pdf}\vspace{-2mm}
		  \caption{Combined measurement using EFTfitter. Corresponding\\ value: $\sigma_{tW} = \SI{24.59 \pm 1.89}{\pico\barn}$}
		\end{minipage}
		%\hfill
		\begin{minipage}[b]{.49\textwidth}
		  \includegraphics[width=1\textwidth]{plots/BLUE_1_poscorr.pdf}\vspace{-2mm}
		  \caption{Combined measurement using BLUE. Corresponding\\ value: $\sigma_{tW} = \SI{24.59 \pm 1.89}{\pico\barn}$}
		\end{minipage}
	\end{figure}\vspace{-2mm}
	\begin{beamercolorbox}[rounded=true,shadow=true]{palette misc}\vspace{-2mm}
		\begin{itemize}
			\item High positive correlation between statistical uncertainties
			%\item Same result for both methods
			\item Both methods behave the same/intended way, uncertainty grows with higher positive correlation
		\end{itemize}
	\end{beamercolorbox}
\end{frame}

\begin{frame}
	\frametitle{}\vspace{-2mm}
	\begin{figure}[h]
	\centering
		\begin{minipage}[b]{.49\textwidth}
		  \includegraphics[width=1\textwidth]{plots/EFT_1_negcorr.pdf}\vspace{-2mm}
		  \caption{Combined measurement using EFTfitter. Corresponding\\ value: $\sigma_{tW} = \SI{25.28 \pm 0.30}{\pico\barn}$}
		\end{minipage}
		%\hfill
		\begin{minipage}[b]{.49\textwidth}
		  \includegraphics[width=1\textwidth]{plots/BLUE_1_negcorr.pdf}\vspace{-2mm}
		  \caption{Combined measurement using BLUE. Corresponding\\ value: $\sigma_{tW} = \SI{25.28 \pm 0.30}{\pico\barn}$}
		\end{minipage}
	\end{figure}\vspace{-2mm}
	\begin{beamercolorbox}[rounded=true,shadow=true]{palette misc}\vspace{-2mm}
		\begin{itemize}
			\item High negative correlation between statistical uncertainties
			%\item Same result for both methods
			\item Both methods behave the same/intended way, uncertainty lowers with higher negative correlation
		\end{itemize}
	\end{beamercolorbox}
\end{frame}


\begin{frame}
	\frametitle{}\vspace{-2mm}
	\begin{figure}[h]
	\centering
		\begin{minipage}[b]{.49\textwidth}
		  \includegraphics[width=1\textwidth]{plots/WORLDEFT.pdf}\vspace{-4mm}
		  \caption{Combined measurement using EFTfitter. Corresponding\\ value: $\sigma_{tW} = \SI{24.64 \pm 4.61}{\pico\barn}$}
		\end{minipage}
		%\hfill
		\begin{minipage}[b]{.49\textwidth}
		  \includegraphics[width=1\textwidth]{plots/WORLDBLUE.pdf}\vspace{-4mm}
		  \caption{Combined measurement using BLUE. Corresponding\\ value: $\sigma_{tW} = \SI{24.64 \pm 4.61}{\pico\barn}$}
		\end{minipage}
	\end{figure}\vspace{-2mm}
	\begin{beamercolorbox}[rounded=true,shadow=true]{palette misc}\vspace{-3mm}
		\begin{itemize}
			\item All 20 uncertainties included with according correlations
			%\item Same result for both methods
			\item Equal result for both methods even with multiple uncertainties
			\item High confidence in the EFTfitter as tool for combination of measurements
		\end{itemize}
	\end{beamercolorbox}
\end{frame}

\section{Constraining the}
\subsection{top quark width}
\begin{frame}
	\frametitle{Procedure for constraining the top quark width} \vspace{-3mm}
	\begin{beamercolorbox}[rounded=true,shadow=true]{palette misc}
		\begin{itemize}
			\item Main objective: calculate a pdf for the top quark width
			\item Build a model with observables, measurements and parameters
			\item Implement dependencies of the observables from the parameters
			\item Include measurements
			\item EFTfitter is used in order to implement that model and perform the calculation
		\end{itemize}\vspace{2mm}
	\end{beamercolorbox}
\end{frame}

\section{Used}
\subsection{Observables}
\begin{frame}
	\begin{minipage}{0.48\textwidth}
	%\begin{beamercolorbox}[rounded=true,shadow=true]{palette misc}
		\begin{itemize}
			\item Cross section of $t\bar{t}$ production $\sigma_{t\bar{t}}$
			\item Cross section of single top production in the t channel $\sigma_{st}$
			\item Branching ratio \texorpdfstring{${BR (t \rightarrow W+b)}$}~
		\end{itemize}
	%\end{beamercolorbox}
	\end{minipage}
	\hfill
	\begin{minipage}{0.50\textwidth}
		\frametitle{The observables}
		\vspace{-1cm}
		\begin{figure}[h]
			\centering
			\begin{subfigure}{.45\textwidth}
			  \centering
			  \includegraphics[width=\linewidth]{images/feyngluonfusiontt2.pdf}
			  %\caption{~\texorpdfstring{$gg \rightarrow t\bar{t} \,$}a  in the t-channel.}
			  \label{fig:ggttt}
			\end{subfigure}
			\begin{subfigure}{.45\textwidth}
			  \centering
			  \includegraphics[width=\linewidth]{images/feyngluonfusionttu.pdf}
			  %\caption{~\texorpdfstring{$gg \rightarrow t\bar{t} \,$}a  in the u-channel.}
			  \label{fig:ggttu}
			\end{subfigure}
			\begin{subfigure}{.45\textwidth}
			  \centering
			  \includegraphics[width=\linewidth]{images/feynqqtt.pdf}
			  %\caption{~\texorpdfstring{$q\bar{q} \rightarrow t\bar{t} \, $}a  in the s-channel.}
			  \label{fig:qqtt}
			\end{subfigure}
			\begin{subfigure}{.45\textwidth}
			  \centering
			  \includegraphics[width=\linewidth]{images/feyngluonfusiontt.pdf}
			  %\caption{~\texorpdfstring{$gg \rightarrow t\bar{t} \, $}a  in the s-channel.}
			  \label{fig:ggtts}
			\end{subfigure}
			\caption{Example processes for \texorpdfstring{$t\bar{t}$}a  production.}
			\label{fig::ttbar}
		\end{figure}
		\vspace{-1cm}
		\begin{figure}[h]
		\centering
			%\begin{subfigure}{.45\textwidth}
			  \centering
			  \includegraphics[width=.5\linewidth]{images/feynst.pdf}
			  \caption{Single top production in the t-channel.}
			  \label{fig:stt}
			%\end{subfigure}
		\end{figure}
	\end{minipage}
\end{frame}

\section{The}
\subsection{Model}
\begin{frame}
\vspace{-1.3cm}
	\frametitle{Overview of the model}
	\begin{table}
		\resizebox{!}{0.11\textwidth}{
		\input{tables/model.tex}}
	\end{table}
	\begin{beamercolorbox}[rounded=true,shadow=true]{palette misc}
	\begin{itemize}
		\item $\Gamma_t$ is added as observable even though no direct measurement will be included 
		\item Relation of $\Gamma_t$ and the top quark pole mass $m_t$:
	\end{itemize}
	\begin{equation*}
		\Gamma = \frac{G_F \, m_t^3}{8 \, \pi \, \sqrt{2}} \left(1 - \frac{M_W^2}{m_t^2} \right)^2 \left(1 + 2 \frac{M_W^2}{m_t^2} \right) \left[1 - \frac{2 \, \alpha_s}{3 \, \pi} \left( \frac{2 \, \pi^2}{3} - \frac{5}{2} \right) \right]
		\label{eqn:gamma}
	\end{equation*}
	\end{beamercolorbox}
\end{frame}

\section{Measurements of}
\subsection{the observables}
\LogoOff
\begin{frame}
\frametitle{The measurements}
\vspace{-0.2cm}
	\begin{beamercolorbox}[rounded=true,shadow=true]{palette misc}
	\begin{itemize}
		\item Measurement for $\sigma_{t\bar{t}}$ from an ATLAS measurement at $\sqrt{s}=\SI{8}{TeV}$:
		\begin{equation*}
			\sigma_{t\bar{t}} = 242.4 \pm 1.7 \, \text{(stat)} \, \pm 10.2 \, \text{(syst)} \, \si{\pico\barn}
		\end{equation*}
		\item The measurement of the inclusive single top quark production cross section in the t-channel from a CMS measurement:
		\begin{equation*}
			\sigma_{st} = 83.6 \pm 2.3 \, \text{(stat)} \, \pm 7.4 \, \text{(syst)} \, \si{\pico\barn}
		\end{equation*}
		\item The measurement of \texorpdfstring{$BR~(t~\rightarrow$~W+b)}~ from a CMS measurement:
		\begin{equation*}
			\text{\texorpdfstring{$BR (t \rightarrow$ W+b)}~} = 1.014 \pm 0.003 \, \text{(stat)} \, \pm 0.032 \, \text{(syst)}
		\end{equation*}
		\item Due to $|V_{tb}| < 1$ an upper limit from that measurement is used:
		\begin{equation*}
			\text{\texorpdfstring{$BR (t \rightarrow$ W+b)}~} = 1.0 + 0.0 - 0.025
		\end{equation*}
	\end{itemize}
	\vspace{-0.2cm}
	\end{beamercolorbox}
\end{frame}
\LogoOn

\section{Model}
\subsection{Implementation}
\begin{frame}
\frametitle{Implementation to EFTfitter}
\vspace{-0.8cm}
\begin{beamercolorbox}[rounded=true,shadow=true]{palette misc}
	\begin{itemize}
		\item Calculation of \texorpdfstring{$BR (t \rightarrow$ W+b)}~ straightforward:
		\begin{equation*}
			\text{\texorpdfstring{$BR (t \rightarrow$ W+b)}~} = \frac{\Gamma(\text{\texorpdfstring{$t \rightarrow$ W+b}~)}}{\Gamma(\text{\texorpdfstring{$t \rightarrow$ W+q}~)}} = |V_{tb}|^2
		\end{equation*}
		\item Calculation of the cross sections with HATHOR (tool to calculate cross section of top quark processes)
		\item HATHOR used with CT10NLO PDF and $\sqrt{s}=\SI{8}{TeV}$
		\item Parameters set to 
		\vspace{-0.3cm}
		\begin{equation*}
		\SI{150}{GeV} < m_t < \SI{200}{GeV} \quad \text{and} \quad |V_{tb}| = \SI{0.55}{}, \SI{0.77}{}, \SI{0.99}{}
		\end{equation*}
		%\vspace{-0.3cm}
		\item Extract dependency of $|V_{tb}|$ from $\sigma_{st}$ by analyzing $\sigma_{st}$ for the three $|V_{tb}|$ values
	\end{itemize}
\end{beamercolorbox}
\end{frame}

\LogoOff
\begin{frame}
\frametitle{HATHOR implementation}
\vspace{-0.4cm}
	\begin{itemize}
		\item Due to computation time HATHOR included as polynomial fit
	\end{itemize}
	\vspace{-0.35cm}
	\begin{figure}
		\begin{subfigure}{0.46\textwidth}
		\includegraphics[width=\linewidth]{plots/ttbar.pdf}
		\end{subfigure}
		\hfill
		\begin{subfigure}{0.46\textwidth}
		\includegraphics[width=\linewidth]{plots/ttbarfit.pdf}
		\end{subfigure}
		\begin{subfigure}{0.46\textwidth}
		\includegraphics[width=\linewidth]{plots/st.pdf}
		\end{subfigure}
		\hfill
		\begin{subfigure}{0.46\textwidth}
		\includegraphics[width=\linewidth]{plots/stvtbfits.pdf}
		\end{subfigure}
	\end{figure}
\end{frame}



\begin{frame}
\frametitle{$|V_{tb}|$ extraction}
	%\begin{beamercolorbox}[rounded=true,shadow=true]{palette misc}\vspace{-2mm}
		\begin{itemize}
			\item Assuming the dependency of $|V_{tb}|$ is quadratic the following extraction is useful
			\begin{equation*}
			\sigma_{st}(m_t,|V_{tb}|) = |V_{tb}|^2 \, \sigma_{st}(m_t)
			\end{equation*}
			\item $\sigma_{st}(m_t)$ follows from mean of "nomalized" distributions (normalized to $|V_{tb}|=1$)
		\end{itemize}
	%\end{beamercolorbox}\vspace{-2mm}
	\vspace{-0.2cm}
	\begin{figure}
			\begin{subfigure}{0.49\textwidth}
			\includegraphics[width=1.1\linewidth]{plots/stvtbfits.pdf}
			\end{subfigure}
			\hfill
			\begin{subfigure}{0.49\textwidth}
			\includegraphics[width=1.1\linewidth]{plots/stvtbfitsnorm.pdf}
			\end{subfigure}
		\end{figure}
\end{frame}

\section{The}
\subsection{Results}
\begin{frame}
\frametitle{The results}
	\begin{itemize}
		\item Providing the model and measurements to the EFTfitter it provides the posterior probabilities
	\end{itemize}
	\begin{figure}
			\begin{subfigure}{0.49\textwidth}
			\includegraphics[width=1.1\linewidth]{plots/resmt.pdf}
			\end{subfigure}
			\hfill
			\begin{subfigure}{0.49\textwidth}
			\includegraphics[width=1.1\linewidth]{plots/resvtb.pdf}
			\end{subfigure}
		\end{figure}
\end{frame}


\begin{frame}
	\begin{itemize}
		\item Comparison between global mode (GM) and world combination (WC) values of the top quark width 
		\begin{equation*}
			\Gamma_t^{GM} = \SI{1.335 \pm 0.040}{GeV} \quad \text{with} \quad  m_t^{GM} = \SI{172.707 \pm 1.554}{GeV}
		\end{equation*}
		\begin{equation*}
			\Gamma_t^{WC} = \SI{1.352 \pm 0.020}{GeV} \quad \text{with} \quad m_t^{WC} = \SI{173.34 \pm 0.76}{GeV}
		\end{equation*}
	\end{itemize}
	\begin{minipage}{0.48\textwidth}
	\begin{itemize}
		\item 2D distribution of the parameters
		\item Top quark width PDF can be calculated from $m_t$ PDF
	\end{itemize}
	\end{minipage}
	\hfill
	\begin{minipage}{0.48\textwidth}
	\begin{figure}
		\centering
		%\includegraphics[page=3,width=\linewidth]{plots/result_plots.pdf}
		\includegraphics[page=3,width=1.1\linewidth]{images/plotres.png}
	\end{figure}
	\end{minipage}
\end{frame}

\begin{frame}
	\begin{minipage}{0.43\textwidth}
		\begin{itemize}
			\item Similarity in shape of $m_t$ and $\Gamma_t$ PDF
			\item Follows from linear dependency visible in plot below
		\end{itemize}
	\end{minipage}
	\hfill
	\begin{minipage}{0.43\textwidth}
		\begin{figure}
			\begin{subfigure}{\textwidth}
			\includegraphics[width=\linewidth]{plots/gammatPDF.pdf}
			\end{subfigure}
		\end{figure}
	\end{minipage}
	\vspace{-0.3cm}
	\begin{figure}
			\begin{subfigure}{0.43\textwidth}
			\includegraphics[width=\linewidth]{plots/resmt.pdf}
			\end{subfigure}
			\hfill
			\begin{subfigure}{0.43\textwidth}
			\includegraphics[page=7,width=\linewidth]{plots/observables.pdf}
			\end{subfigure}
		\end{figure}
\end{frame}
\LogoOn

\section{The}
\subsection{End}
\begin{frame}
\frametitle{Conclusions}
	\begin{beamercolorbox}[rounded=true,shadow=true]{palette misc}
	\begin{itemize}
		\item Combination of measurements works reliably
		\item Global mode values agree within one standard deviation
		\item Resulting plots fit expectations
		\item Possible next steps
		\begin{itemize}
			\item Expand the model with more observables
			\item Add more (also multiple of the same) measurements
			\item Create a "global fit" on physical quantities that cannot be measured directly
		\end{itemize}
		\item Thanks for your attention
	\end{itemize}
	\end{beamercolorbox}
\end{frame}


\newcommand{\backupbegin}{
   \newcounter{framenumberappendix}
   \setcounter{framenumberappendix}{\value{framenumber}}
}
\newcommand{\backupend}{
   \addtocounter{framenumberappendix}{-\value{framenumber}}
   \addtocounter{framenumber}{\value{framenumberappendix}} 
}

\appendix
\backupbegin
\section{Appendix}
\subsection{}

\begin{frame}
\frametitle{Observables}
	\begin{figure}
		\centering
		\includegraphics[width=0.8\textwidth,page=1]{plots/observables.pdf}
	\end{figure}
\end{frame}

\begin{frame}
\frametitle{Observables}
\begin{minipage}{0.49\textwidth}
	\begin{figure}

		\includegraphics[width=1.1\textwidth,page=3]{plots/observables.pdf}
	\end{figure}
\end{minipage}
\hfill
\begin{minipage}{0.49\textwidth}
	\begin{figure}

		\includegraphics[width=1.1\textwidth,page=4]{plots/observables.pdf}
	\end{figure}
\end{minipage}
\end{frame}

\begin{frame}
\frametitle{Observables}
	\begin{figure}
		\centering
		\includegraphics[width=0.8\textwidth,page=7]{plots/observables.pdf}
	\end{figure}
\end{frame}

\LogoOff
\begin{frame}
\frametitle{Single top measurement correlation}
	\begin{figure}
		\centering
		\includegraphics[width=0.7\textwidth]{plots/singletopcorrres.png}
	\end{figure}
\end{frame}
\LogoOn

\backupend
\end{document}